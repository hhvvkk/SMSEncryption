\section{Introduction}

\subsection{Purpose}%\subsection{Background}
This document describes the software requirements and specification for the SMSEncryption mobile application.
\vspace{10pt}\\
The document will be used to ensure requirements are well understood by all stakeholders. It is therefore intended for all stakeholders of the project including the developers and customers.


\subsection{Background}
Reliable communication in certain parts of South Africa is not always possible in remote locations using GSM, 3G or other similar mediums.
\vspace{10pt}\\
Therefore, communication normally occur using SMS which generally is not very secure. This can cause a loss in confidentiality, integrity and availability of the communicators.
\vspace{10pt}\\
There is a need to develop a secure way of communicate using conventional mediums such as SMS.

\subsection{Scope}
%objectives, goals, benefits, 
%Relate the software to corporate goals or business strategies.
The goal of this project is to create a mobile application which can be used on more than one platform(i.e. IOS and Android). This application will be able to encrypt messages which can then be decrypted on the receiving end.
\vspace{10pt}\\
By using SMSEncryption, the user will be able to encrypt messages which can only be decrypted using the same application. The user will require local authentication to access the access the appliaction and make use of it's features.
\vspace{10pt}\\
The benefit of this application is that you can use SMS technology to send messages containing confidential information which only you and the desired recipient can read in an unencrypted form.


\subsection{Definitions, acronyms and abbreviations}
\begin{itemize}
\item SMSEncryption - The current project which will allow you to encrypt and decrypt text with the purpose of it being sent as a message via messaging applications eg. WhatsApp, SMS etc.
\item Message - The text intended to be sent from a sender to a receiver  or stored once said message has been encrypted via SMSEncryption.
\item Plaintext -  Is information a sender wishes to transmit to a receiver. 
\item Ciphertext - Is the result of encryption performed on plaintext using an algorithm, called a cipher.
\item Encrypt -  To alter the plaintext using an algorithm so as to be unintelligible to unauthorized parties.
\item Decrypt - The act of decoding a ciphertext back into the orginal form before conversion took place.
\item User - An authorised person who will interact with the application.
\item Sender - The person who authored and intends to send a messagethat has been encrypted via the application.
\item Receiver - The intended party who receives a message which has been encrypted via the application.
\item SMS - Short Message Service (SMS) is a text messaging service component of phone, Web, or mobile communication systems. This allows for short messages to be sent to other devices over a network which is not controlled by the sender or receiver.
\item GSM - Global System for Mobile Communications (GSM) is a second generation standard for protocols used on mobile devices.
\item Entropy -  The expected value of the information contained in a message.
\end{itemize}

\subsection{Document Conventions}
\begin{itemize}
\item Documentation formulation: LaTeX
\item Naming convention: Crows Foot Notation
\end{itemize}

\subsection{References}
\begin{itemize}
\item{Kyle Riley - MWR Info Security}
\begin{itemize}
\item face-to-face meeting
\item email
\end{itemize}

\item{Bernard Wagner - MWR Info Security}
\begin{itemize}
\item face-to-face meeting
\item email
\end{itemize}
%\item{websites???}
\end{itemize}

\subsection{Overview}
%Structure of the rest of the SRS, in particular:
%organization for Section 3 (Specific Requirements)
%deviations from the standard SRS format
The rest of the document will be organized to include General Description and Specific Requirements for SMSEncryption application.
\vspace{10pt}\\
The General Description will provide a background to the reader for SMSEncryption and contains sections: Product perspective, Product functions, User characteristics, Constraints and Assumptions and dependencies.
\vspace{10pt}\\
The Specific requirements contain requirements for SMSEncryption and is organised by features. This is done in such a way which will highlight the functions of the application. 
\vspace{10pt}\\
The sections contained in Specific requirements include External interface requirements, System Features, Performance Requirements, Design constraints, Software system attributes, and Other requirements.
\vspace{10pt}
%Appendices consist also of minutes of customer interviews
%and meetings, and do not constitute additional
%requirements of the software; i.e., all requirements arising
%from these minutes have been incorporated into the
%specific requirements in Section 3


