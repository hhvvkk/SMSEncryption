\section{Introduction}

\subsection{Purpose}%\subsection{Background}
This document describes the software requirements and specifications for the SMSEncryption mobile application.
\vspace{10pt}\\
The document will be used to ensure requirements are well understood by all stakeholders. It is therefore intended for all stakeholders of the project; including the developers and customers.


\subsection{Background}
Certain operations within remote parts of South Africa require reliable transmission of data which cannot be achieved through traditional means of data transmission (e.g. GSM, 3G, etc). As an alternative, SMSes are used to transmit these messages. Some of the data which is transmitted is sensitive, and requires some form of encryption. As traditional encryption functions often produce characters which fall outside the character set of modern cellphones, it is necessary to use an encoding scheme to translate these characters back into the cell phones character set. Most encoding schemes (such as base64) increase the length of the message this may result in the encrypted message exceeding the maximum character allowance for SMS.

\subsection{Scope}
%objectives, goals, benefits, 
%Relate the software to corporate goals or business strategies.
The goal of this project is to create a mobile application which can be used to encrypt text before sending it via SMS technology, which can be decrypted on the receiving end. The application must be able to be used on more than one platform (i.e. iOS and Android).
\vspace{10pt}\\
By using the SMSEncryption application, the user will be able to encrypt messages which can only be decrypted by using the same application on the receiving end. The application will require local authentication in order to gain access to the appliaction and make use of it's features.
\vspace{10pt}\\
The benefit of this application is that you can use SMS technology to send confidential messages which can only be viewed by you and the desired recipient of the message - who is the only party who can unencrypt the message.

%Perhapts put this section below, higher up in the document, as some of these abbreviations are already used in the text above?
\subsection{Definitions, acronyms and abbreviations}
\begin{itemize}
\item SMSEncryption - The name of the current project which will allow users to encrypt and decrypt text with the main purpose of it being sent as an SMS, or via other messaging applications such as WhatsApp, WeChat, Facebook chat etc.
\item Message - The text intended to be sent from a sender to a receiver  or stored once said message has been encrypted via SMSEncryption.
\item Plaintext -  Is information a sender wishes to transmit to a receiver. 
\item Ciphertext - Is the result of encryption performed on plaintext using an algorithm, called a cipher.
\item Encrypt -  To alter the plaintext using an algorithm so as to be unintelligible to unauthorized parties.
\item Decrypt - The act of decoding a ciphertext back into the orginal form before conversion took place.
\item User - An authorised person who will interact with the application.
\item Sender - The person who authored and intends to send a messagethat has been encrypted via the application.
\item Receiver - The intended party who receives a message which has been encrypted via the application.
\item SMS - Short Message Service (SMS) is a text messaging service component of phone, Web, or mobile communication systems. This allows for short messages to be sent to other devices over a network which is not controlled by the sender or receiver.
\item SMSC - Short Message Service Centre (SMSC) is a network element in the mobile telephone network. Its purpose is to store, forward, convert and deliver SMS messages.
\item GSM - Global System for Mobile Communications (GSM) is a second generation standard for protocols used on mobile devices.
\item Entropy -  The expected value of the information contained in a message.
\end{itemize}

\subsection{Document Conventions}
\begin{itemize}
\item Documentation formulation: LaTeX
\item Naming convention: Crows Foot Notation
\end{itemize}

\subsection{References}
\begin{itemize}
\item{Kyle Riley - MWR Info Security}
\begin{itemize}
\item face-to-face meeting
\item email
\end{itemize}

\item{Bernard Wagner - MWR Info Security}
\begin{itemize}
\item face-to-face meeting
\item email
\end{itemize}
%\item{websites???}
\end{itemize}

\subsection{Overview}
%Structure of the rest of the SRS, in particular:
%organization for Section 3 (Specific Requirements)
%deviations from the standard SRS format
The rest of the document will be organized to include the following sections: General Description and Specific Requirements for the SMSEncryption application.
\vspace{10pt}\\
The General Description section will provide a background to the reader for the SMSEncryption application, and contains the following sections: Product Perspective, Product Functions, User Characteristics, Constraints and Assumptions and Dependencies.
\vspace{10pt}\\
The Specific Requirements section contains requirements for the SMSEncryption application, and is organised by application features. This is done in such a way that it will highlight the functions of the application. 
\vspace{10pt}\\
The sections contained in Specific Requirements include External Interface Requirements, System Features, Performance Requirements, Design Constraints, Software System Attributes, and Other Requirements.
\vspace{10pt}
%Appendices consist also of minutes of customer interviews
%and meetings, and do not constitute additional
%requirements of the software; i.e., all requirements arising
%from these minutes have been incorporated into the
%specific requirements in Section 3


