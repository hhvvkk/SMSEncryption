\section{Appendix B - One time pads}

%Note:: Die * is om die subsections uit die 
%content page uit te hou. Dit unclutter dit so `n betjie
\subsection*{Introduction}

This is about research done into one time pads, an encryption technique that if used correctly is unbreakable.  It also provides the person attempting to decrypt the message with no information about the plaintext apart from the max possible length it could be.
\subsection*{Method}

We did some research into one time pads and why it is that they are so strong. After that we implemented a onetime pad algorithm and it looked very promising.
\subsection*{Result}

The encryption is very strong, allows for 1 to 1 character encryption thus enabling us to have a plain text message of 160 characters fully utilizing space. It seemed to be the solution to the problem.
\subsection*{Discussion}

The first thing that comes to mind when thinking about one time pad encryption is how to distribute the pad. The pad needs to be distributed between the two parties and the must any given moment in time know what the next �line� that will be used will be, in other words it requires synchronization.
\subsection*{Conclusion}

In terms of message length and encryption strength it is perfect but with no way of distrusting the one time pad securely we had to disregard this solution.
\subsection*{References}

Electronic, M., n.d. One Time Pad Encryption,The unbreakable encryption method. s.l.:mils electronic gesmbh & cokg.
