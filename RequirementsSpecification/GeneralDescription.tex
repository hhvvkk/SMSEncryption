\section{General description}

\subsection{Product perspective}
 %Describes the environment of the system

\subsubsection{Description}
SMSEncryption is a new product which can be used in conjunction with any mobile text manipulation application, such as the general keyboard input used by the different mobile operating systems, or any other text manipulating application, capable of using the basic GSM character set - should you require encryption and decryption functionality for secure communication between two parties.
\vspace{12pt}\\
The GSM character set contains a limited amount of characters, which will, in turn, limit the encryption methods we can make use of, as many encryption algorithms greatly increase both the size of the message, and the number of different characters used. Making use of these encryption algorithms that generate large amounts of characters will be infeasible, as sending large SMSes will be expensive.
\vspace{12pt}\\
Software interface - The software interface will make use of operating system features, such as a clipboard on the device to facilitate 'copying' and 'pasting' of texts or ciphertexts.
\vspace{12pt}\\
User interface - The user interface is what will allow the user to type a message, encrypt it, copy the ciphertext, and paste it into the application that will send the message. On the receiving end, the message received will be copied, and pasted into the SMSEncryption application, which will be used to decrypt the received message. This ensures integrity of the message, as only users of the application will be able to encrypt/decrypt the message in the agreed upon way.
\vspace{12pt}\\
Hardware Interface - The software will run on a mobile device that allows user interaction and text manipulation.


%Note:
%Required : Functionality must be provided if service is provided
%Extends : Extended functionality which can be provided, but not always.
\newpage
\subsubsection{Use Cases}
SMSEncryption Use case diagram

\begin{center}
 \includegraphics[width=13cm]{diagrams/UseCaseDiagrams/UsecaseV4.png}
\end{center}

\newpage
\subsubsection{State Diagram}
SMSEncryption State diagram

\begin{center}
 \includegraphics[width=13cm]{diagrams/StateDiagrams/SMSEncryptionStateMachine.png}
\end{center}


\subsection{Product features}
%An overview of the system's main features
%? complete list of "brief" use-case descriptions
%? features will be specified in detail in Section 3
%check for the "lock time" when password is typed incorrectly
%check first item for cipher message list
\subsubsection{Log In}
\begin{itemize}
\item On first use of the application, a user account with a password must be created that will ensure user authentication.
\item Every time a user wants to use the application, the password for that user must be provided along with that user's username.
\item If the provided password (and related details) are entered correctly, the user gains access to the application.
\item If the password is entered incorrectly after a specified number of tries, the application will lock for a specified amount of time - preventing access from an unauthorised user; effectively nullifying "guessing" attacks.
\end{itemize}
\subsubsection{Message}
\textbf{Create}
\begin{itemize}
\item The plaintext is created independently by the user and input into the application.
\item The plaintext can also be created and edited within the application.
\end{itemize}
\textbf{Cipher Message}
\begin{itemize}
\item The user will select a relevant contact, that contact's set keys will then be used to perform encryption or decryption.
\item The message will be encrypted to obtain the cipher text, which the user can then copy out and paste into the application that will send the ciphertext to the desired receiver; this can be done via any messaging application.
\item The desired receiver will be able to decrypt the message back into its original plaintext.
\end{itemize}

\subsubsection{Device Synchronization}
\begin{itemize}
\item In order for communication to take place between two devices they need to be synchronized.
\item A user adds what is called a "contact". The SMSEncryption application will require the name of the contact, a unique key for the current user of the application, as well as the unique key from the contact to be added. The unique keys are easily generated within the application.
\item Both users must add each other at the same time, because their unique (secret) keys need to be shared with each other to be able to communicate with each other. This will synchronize communication between the two contacts/devices.
\item Once a contact has been added, you can resynchronise with that contact at any time in the future should it be needed. This will be required if users accidentally modify their keys, or if synchronization errors take place.
\end{itemize}

\subsection{User characteristics}
%Assumptions about the users, their background, how
%much training they will need
%? e.g., different user interfaces for expert vs. novice users
%? Only user characteristics that affect the software
%requirements
\begin{itemize}
\item * Local authentication will ensure that only the intended user of the application will have full access to the application features and data.
\item It is assumed that the user has proficient knowledge on how to copy text from messages, such as SMSes, and paste it within this application.
\item It is also assumed that users have performed the device synchronization phase correctly before any communication takes place, as the application will not do this automatically, and it is required for the application to function correctly.
\end{itemize}


\subsection{Constraints}
%Anything that will limit the designer's options
\begin{itemize}
\item The application must make use of the basic GSM character set.
\end{itemize}



\subsection{Assumptions and dependencies}
%List any assumed factors (as opposed to known facts) 
%that could affect the requirements stated in the SRS. 

\begin{itemize}
\item It is assumed that the amount of characters in the basic GSM character set is 128 for the 7-bit encoding used in GSM.
\item It is assumed that the devices being used allows for copying and pasting of text between different interfaces and applicaitons.
\end{itemize}
